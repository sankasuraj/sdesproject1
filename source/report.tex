\documentclass[12pt, a4paper]{report}

\usepackage{graphicx}
\usepackage{authblk}
\usepackage{hyperref}
\usepackage{float}

\title{Pendulum with Friction}
\author{Sanka Venkata Sai Suraj}
\affil{Rollno: 130010057}
\date{\today}

\begin{document}
\maketitle

\begin{abstract}
The problem in hand is to simulate the movement of a pendulum with friction. For this a mass-less pendulum with a bob at it's end is chosen. The Ipython notebook will generate an animation showing the motion of the pendulum using matplotlib and the python script will also generate a plot which will contain $\dot{\theta}$ and $\theta$ plots with respect to time. The python code generated these plots by taking the mass, length and damping constant of the pendulum and $\dot{\theta}$ and $\theta$ as arguments for the function and these can be changed which will automatically reflect in the pdf report once the make file in run.
\end{abstract}

\begin{itemize}
\item Public git repository with open source code can be found at \url{https://github.com/sankasuraj/sdesproject1}
\item Ipython2 notebook version 1.2.1 or higher is required to run the Ipython code
\item Python 2.7 version is required to run the Python code
\item Numpy with version 1.8.2 or higer is required
\item Matplotlib with version 1.3.1 or higher is required
\item A media player is required to play the mp4 animation created using the ipython notebook \cite{ref1}
\end{itemize}

\section*{Equation of Motion}
Equation of motion for a mass-less pendulum with a bob at the end is 

\begin{equation}
\label{equation1}
\ddot{\theta} + \frac{b}{m}\dot{\theta} + \frac{g}{L}sin\theta = 0 \cite{ref2}
\end{equation}
\noindent
where,\\
b = damping constant in $\frac{kg}{s}$\\
m = mass of the bob in $kg$\\
g = acceleration due to gravity which is taken to be $9.8\frac{m}{s^2}$\\
L = length of the pendulum in $m$\\
$\theta$ = Angular displacement of the pendulum from the vertical in $rad$\\
$\dot{\theta}$ = Angular velocity of the pendulum in $\frac{rad}{s}$\\
$\ddot{\theta}$ = Angular acceleration of the pendulum in $\frac{rad}{s^2}$\\
\noindent
In equation \ref{equation1} using small angle approximation the equation can be re-written as

\begin{equation}
\label{equation2}
\ddot{\theta} + \frac{b}{m}\dot{\theta} + \frac{g}{L}\theta = 0
\end{equation}

\section*{Solution}
The equation can be solved assuming the general solution to be $\theta = e^{\lambda}$
Substituting this in equation \ref{equation2} we get

\begin{eqnarray}
\lambda^2 + \frac{b}{m}\lambda + \frac{g}{L} = 0\\
\lambda_1 = \frac{-\frac{b}{m} + \sqrt{(\frac{b}{m})^2 - \frac{4g}{L}}}{2}\\
\lambda_2 = \frac{-\frac{b}{m} - \sqrt{(\frac{b}{m})^2 - \frac{4g}{L}}}{2}\\
\end{eqnarray}

\noindent
Solution can be written as 
\begin{equation}
\label{equation3}
\theta = c_1e^{\lambda_1} + c_2e^{\lambda_2}
\end{equation}

\noindent
Using initial conditions for $\theta$ and $\dot{\theta}$, constants $c_1$ and $c_2$ in equation \ref{equation3} can be written as
\begin{eqnarray}
c_1 = \frac{\dot{\theta}(0) - \lambda_2\theta(0)}{\lambda_1 - \lambda_2}\\
c_2 = \frac{\lambda_1\theta(0) - \dot{\theta}(0)}{\lambda_1 - \lambda_2}
\end{eqnarray}

\noindent
This solution will work for both underdamped and overdamped cases. It won't work for critical damped case because $\lambda_1=\lambda_2$ in critical damped case. However for practical values of damping constant `b' the pendulum will always be in underdamped condition so the critical damped case is ignored where the above solution won't work.

\section*{Results}
Plot of $\theta$ and $\dot{\theta}$ with respect to time for the pendulum with friction is:\\
\begin{figure}[H]
\label{pendulum}
\includegraphics[width = \textwidth]{../output/pendulum.png}
\caption{$\dot{\theta}$ and $\theta$ w.r.t time}
\end{figure}

\bibliographystyle{plain}
\bibliography{Ref}

\end{document}